\documentclass[t,% Place text of slides at the (vertical) top of the slides
brazilian,% Brazilian Portuguese, FTW!
11pt,% Standard font size
aspectratio=169,% Aspect ratio 16:9 (widescreen)
table% xcolor option
]{beamer}

\setbeamercolor{footnote mark}{fg=red}

\usetheme{Boadilla}
\setbeamertemplate{navigation symbols}{}
\setbeamertemplate{frametitle continuation}{}
\setbeamertemplate{page number in head/foot}[framenumber]
\setbeamertemplate{enumerate items}[default]
\setbeamertemplate{itemize items}[circle]
\setbeamercovered{transparent}
\setbeamerfont{frametitle}{size=\normalsize}

\usepackage{babel}
\usepackage[utf8]{inputenc}
\usepackage[T1]{fontenc}
\usepackage{lmodern}

\usepackage{graphicx}
\setkeys{Gin}{keepaspectratio}

\usepackage{amssymb,amsfonts,amsmath}
\usepackage{mathtools}

\usepackage{siunitx}
\sisetup{locale = FR}

\usepackage{tikz}
\usetikzlibrary{calc}

\usepackage{pgffor}
\usepackage{etoolbox}

% \usepackage{colortbl}

\usepackage{tcolorbox}

\usepackage{pgfplots}
\pgfplotsset{compat=1.18}

\newcommand{\esima}{\textordfeminine }
\newcommand{\esimo}{\textordmasculine }

\newcommand{\vboxcorr}[2]{%
    \resizebox{!}{\totalheight-#1}{#2}%
}
\newcommand{\hboxcorr}[2]{%
    \resizebox{\width-#1}{!}{#2}%
}

\DeclareMathOperator{\arctg}{arctg}
\DeclareMathOperator{\sen}{sen}
\DeclareMathOperator{\arcsen}{arc sen}

\def\Disciplina{Física Experimental III}
\def\Professor{Rodrigo de Farias Gomes}
\def\Periodo{Período 2025.1}

\title{\Disciplina}
\author{\Professor}
\date{\Periodo}

\begin{document}

\begin{frame}
    \titlepage
\end{frame}

\begin{frame}{Informações gerais}
    \begin{itemize}
        \item Nome: {\fontfamily{augie}\selectfont Rodrigo de Farias Gomes}
        \item Telefone (somente mensagens): (92) 9 9405-1724
        \item E-mail: shpnft@gmail.com
        \item Sala: 201b, Bloco E (2\esimo{} pavimento)
    \end{itemize}
\end{frame}

\begin{frame}<1>[label=ementa]{Ementa de \Disciplina}
    \begin{enumerate}
        \item Resistores lineares e não lineares
        \item Corrente elétrica induzidas
        \item Curva de carga de um capacitor
        \item Lei de Ohm e resistividade elétrica
        \item Circuitos elétricos e Leis de Kirchoff
        \item Efeito Joule e efeito termoelétrico
        \item Demonstração da Força de Lorentz
        \item Interações entre campos magnéticos e correntes elétricos
    \end{enumerate}
\end{frame}

\begin{frame}{Avaliação}
    \begin{itemize}
        \item A avaliação será na forma de 8 notas, sendo que a média dos 
            exercícios escolares (\(ME\)) será dada por
            \[
                ME=\frac{N_1+N_2+N_3+N_4+N_5+N_6+N_7+N_8}{8}
            \]
        \item Se \(MEE \geq 8,0\), então a média final (\(MF\)) será igual à \(MEE\)
        \item Se \(MEE < 8,0\), então
            \[
                MF=\frac{2\times MEE+PF}{3}
            \]
            onde PF é a nota da \textbf{prova final}
        \item Se \(MF \geq 5,0\) e a frequência em sala for maior que 75\%, o aluno está aprovado
    \end{itemize}
\end{frame}

\begin{frame}{Aulas}
    \begin{columns}
        \begin{column}{0.45\textwidth}
            \begin{enumerate}
                \item 17/03/25
                \item 24/03/25
                \item 31/03/25
                \item 07/04/25
                \item 14/04/25
                \item 28/04/25
                \item 05/05/25
                \item 12/05/25
                \item 19/05/25
                \item 26/05/25
            \end{enumerate}
        \end{column}

        \begin{column}{0.45\textwidth}
            \begin{enumerate}\setcounter{enumi}{10}
                \item 02/06/25
                \item 09/06/25
                \item 16/06/25
                \item 23/06/25
                \item 30/06/25
                \item 07/07/25
                \item 14/07/25
            \end{enumerate}
        \end{column}
    \end{columns}
\end{frame}

\againframe<2>{ementa}

\begin{frame}{Atividade 1}
    \textit{A definir}
\end{frame}
\end{document}
