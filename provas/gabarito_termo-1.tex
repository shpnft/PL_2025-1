\documentclass[12pt,a4paper,brazilian]{article}

\usepackage{babel}
\usepackage[utf8]{inputenc}
\usepackage[T1]{fontenc}
\usepackage{lmodern}

\usepackage{amssymb,amsfonts,amsmath}

\usepackage{tikz}
\usetikzlibrary{calc,intersections}

%https://tex.stackexchange.com/a/100406
%29.7 cm - 1cm - 1cm - 144.90/28.4 cm = 22.60 cm
\usepackage[a4paper, totalheight=22.60cm,includeheadfoot,left=1.5cm, right=1.0cm, top=1cm]{geometry}
\setlength{\headheight}{144.90pt}

\setkeys{Gin}{keepaspectratio}

\newcommand{\cabeca}{
    \begin{tikzpicture}
        \node(Logo) {\includegraphics[width=2.5cm]{logo.png}};
        % \node(Logo) {\includegraphics[width=4.1cm]{logo.png}};

        \node(Local) at (Logo.north east) [anchor=north west, yshift=-0.25cm,
            align=center, execute at begin node=\setlength{\baselineskip}{3ex}]
            {
                \huge{\textbf{Universidade Federal do Amazonas}} \\
                \large{\textbf{Instituto de Ciências Exatas e Tecnologia}} \\
                \large{\textbf{\Description}}
            };

        \node(Ident) at (Local.south west) [anchor=north west, yshift=-0.25cm,
            align=left, execute at begin node=\setlength{\baselineskip}{2em}]
            {
                Professor: {\fontfamily{augie}\selectfont \Professor} \\
                Aluno(s):
            };
        % \draw [thick] (Logo.south west) -- ($(Logo.south west -| Local.south east)$);
        % \draw [red] (Logo.north west) rectangle (Logo.south east);
        % \draw [blue] (Local.north west) rectangle (Local.south east);
        % \draw [green] (Ident.north west) rectangle (Ident.south east);
    \end{tikzpicture}
}

\usepackage{fancyhdr}
\fancyhead{}
\fancyfoot{}
\fancyhead[c]{\cabeca}
\fancyfoot[r]{\fontfamily{augie}\selectfont Boa sorte!}

\pagestyle{fancy}
\renewcommand{\headrulewidth}{0pt}
\renewcommand{\footrulewidth}{0pt}

\newcommand{\ratio}[1]{(#1\% da nota)}
%-----------------------------------CUT HERE-----------------------------------

\def\Description{Termodinâmica -- Prova 1}
\def\Professor{Rodrigo de Farias Gomes}

\usepackage{siunitx}
\sisetup{locale = FR}

\usepackage{tcolorbox}
\tcbset{boxrule=0pt, top=0pt, bottom=0pt}

\DeclareMathOperator{\sen}{sen}
\DeclareMathOperator{\tg}{tg}
\usepackage{enumitem}

\begin{document}

\begin{tcolorbox}[colback=black!10, colframe=black!50, title=Observações]
    \begin{itemize}
        \item Todas as páginas com resposta devem ter o nome e matrícula do
            aluno escritos com caneta no início (cabeçalho) ou no final
            (rodapé). Páginas que não obedeçam a esse critério não serão usadas
            na avaliação
    \end{itemize}
\end{tcolorbox}

\vspace{2em}

\begin{enumerate}
    \item\ratio{30} A equação de estado de um certo gás é 
        \[
            (P+b)v = RT
        \]
        e sua energia interna específica é dada por 
        \[
            u=aT+bv+u_0
        \]
        \begin{enumerate}
            \item Encontre \(c_v\)
            \item Mostre que \(c_P -c_v = R\)
        \end{enumerate}
    \item\ratio{30} Em um só diagrama \(T-S\), esboce curvas para os seguintes processos
        reversíveis para um gás ideal, começando do mesmo estado inicial: (a)
        uma expansão isotérmica, (b) uma compressão adiabática, (c) uma expansão
        isobárica e (d) um processo isocórico em que é adicionado calor.

    \item\ratio{40} Uma massa \(m\) de um líquido a uma temperatura \(T_1\) é misturada
        com uma massa \(3m\) do mesmo líquido a uma temperatura \(T_2\).
        O sistema está termicamente isolado. Mostre que a variação de entropia do
        universo é
        \[
            4 m c_P \ln\frac{\tau+3}{4\tau^{1/4}}
        \]
        onde \(\tau=T_1/T_2\).


\end{enumerate}

\newpage
\def\Description{Termodinâmica -- Gabarito}

\begin{enumerate}
    \item%
        \begin{enumerate}
            \item Temos que
                \[
                    c_v = \left(\frac{\partial u}{\partial T}\right)_v \implies
                    c_v = \left(\frac{\partial}{\partial T}(aT+bv+u_0)\right)_v = a
                \]
            \item Temos que
                \[
                    c_P - c_v = \left[\left(\frac{\partial u}{\partial v}\right)_T+P\right]
                    \left(\frac{\partial v}{\partial T}\right)_P
                \]

                Isolando \(v\), temos
                \[
                    v = \frac{RT}{P+b} \implies 
                    \left(\frac{\partial v}{\partial T}\right)_P = 
                    \frac{R}{P+b}
                \]

                Como
                \[
                    \left(\frac{\partial u}{\partial v}\right)_T = b
                \]
                podemos encontrar que
                \[
                    c_P -c_v = (b+P)\frac{R}{P+b} = R
                \]
        \end{enumerate}
    \item%
        \begin{enumerate}   
            \item Para um gás ideal, a entropia de um processo isotérmico é dada por
                \[
                    s_1 - s_0 = nR \ln\frac{v_1}{v_0}
                \]
                de forma que, numa expansão, \(s_1 > s_0\).
            \item Num processo adiabático \(s_1 = s_0\) e \(Tv^{\gamma - 1}=\text{constante}\), de forma que
                numa compressão (\(v\downarrow\)) temos que a temperatura aumenta enquanto que numa
                expansão (\(v\uparrow\)) temos que a temperatura diminui
            \item Para um gás ideal, temos que \(Pv=RT\) e para um processo isobárico
                \[
                    s_1 - s_0 = c_P \ln\frac{T_1}{T_0} \implies
                    T_1 = T_0 \exp\left(\frac{s_1-s_0}{c_P}\right)
                \]
                Numa expansão, \(v\) aumenta e, portanto, \(T\) aumenta
            \item De forma análoga, para um processo isocórico
                \[
                    s_1 - s_0 = c_v \ln\frac{T_1}{T_0} \implies
                    T_1 = T_0 \exp\left(\frac{s_1-s_0}{c_v}\right)
                \]
                Temos que \(\Delta E = Q+W\) onde, para um gás ideal, \(\Delta E \propto \Delta T\) e,
                para um processo isocórico, \(W=0\). Dessa forma, \(\Delta T \propto Q\) e a temperatura 
                aumenta quando o sistema recebe calor.

                É importante notar que \(c_P > c_v\) e, portanto, a curva do processo isocórico
                será mais inclinada do que a do processo isobárico.
        \end{enumerate}
        Finalmente
        \begin{center}
            \begin{tikzpicture}
                \draw [thick,->] (0,0) -- (5,0) node [right] {\(S\)};
                \draw [thick,->] (0,0) -- (0,5) node [left] {\(T\)};
                \draw (1,1) -- (4,1) node [right] {(a)};
                \draw (1,1) -- (1,4) node [above] {(b)};
                \draw [domain=1:4, samples=200] plot ({\x},{exp(\x*ln(4)/3)/pow(4,1/3)}) node [below right] {(c)};
                \draw [domain=1:3, samples=200] plot ({\x},{exp(\x*ln(4)/2)/pow(4,1/2)}) node [above left] {(d)};
            \end{tikzpicture}
        \end{center}

    \item A variação de entropia de cada ''porção'' de água é dada por
        \begin{gather}
            \Delta S_1 = m c_P \ln\frac{T_f}{T_1} \\
            \Delta S_2 = 3m c_P \ln\frac{T_f}{T_2} 
        \end{gather}
        e a variação de entropia do universo é dada por 
        \[
            \Delta S_\text{tot} = \Delta S_1 + \Delta S_2 = 
            m c_P \ln\frac{T_f}{T_1} + 3m c_P \ln\frac{T_f}{T_2} \implies
            \Delta S_\text{tot} = mc_P\ln\left(\frac{T_f^4}{T_1 T_2^3}\right)
        \]

        A temperatura \(T_f\) pode ser encontrada através das trocas de calor
        \[
            T_f = \frac{mT_1 + 3mT_2}{4m} = \frac{T_1+3T_2}{4}=T_2 \frac{\tau + 3}{4}
        \]
        Substituindo e simplificando, temos
        \[
            \Delta S_\text{tot} = 4 mc_P \ln\left(\frac{\tau+3}{4\tau^{1/4}}\right)
        \]

\end{enumerate}
\end{document}
