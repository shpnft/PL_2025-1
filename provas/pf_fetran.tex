\documentclass[12pt,a4paper,brazilian]{article}

\usepackage{babel}
\usepackage[utf8]{inputenc}
\usepackage[T1]{fontenc}
\usepackage{lmodern}

\usepackage{amssymb,amsfonts,amsmath}

\usepackage{tikz}
\usetikzlibrary{calc,intersections}

%https://tex.stackexchange.com/a/100406
%29.7 cm - 1cm - 1cm - 144.90/28.4 cm = 22.60 cm
\usepackage[a4paper, totalheight=22.60cm,includeheadfoot,left=1.5cm, right=1.0cm, top=1cm]{geometry}
\setlength{\headheight}{144.90pt}

\setkeys{Gin}{keepaspectratio}

\newcommand{\cabeca}{
    \begin{tikzpicture}
        \node(Logo) {\includegraphics[width=2.5cm]{logo.png}};
        % \node(Logo) {\includegraphics[width=4.1cm]{logo.png}};

        \node(Local) at (Logo.north east) [anchor=north west, yshift=-0.25cm,
            align=center, execute at begin node=\setlength{\baselineskip}{3ex}]
            {
                \huge{\textbf{Universidade Federal do Amazonas}} \\
                \large{\textbf{Instituto de Ciências Exatas e Tecnologia}} \\
                \large{\textbf{\Description}}
            };

        \node(Ident) at (Local.south west) [anchor=north west, yshift=-0.25cm,
            align=left, execute at begin node=\setlength{\baselineskip}{2em}]
            {
                Professor: {\fontfamily{augie}\selectfont \Professor} \\
                Aluno(s):
            };
        % \draw [thick] (Logo.south west) -- ($(Logo.south west -| Local.south east)$);
        % \draw [red] (Logo.north west) rectangle (Logo.south east);
        % \draw [blue] (Local.north west) rectangle (Local.south east);
        % \draw [green] (Ident.north west) rectangle (Ident.south east);
    \end{tikzpicture}
}

\usepackage{fancyhdr}
\fancyhead{}
\fancyfoot{}
\fancyhead[c]{\cabeca}
\fancyfoot[r]{\fontfamily{augie}\selectfont Boa sorte!}

\pagestyle{fancy}
\renewcommand{\headrulewidth}{0pt}
\renewcommand{\footrulewidth}{0pt}

\newcommand{\ratio}[1]{(#1\% da nota)}
%-----------------------------------CUT HERE-----------------------------------

\def\Description{Fenômenos de Transporte -- Prova Final}
\def\Professor{Rodrigo de Farias Gomes}

\usepackage{siunitx}
\sisetup{locale = FR}

\usepackage{tcolorbox}
\tcbset{boxrule=0pt, top=0pt, bottom=0pt}

\DeclareMathOperator{\sen}{sen}
\DeclareMathOperator{\tg}{tg}
\usepackage{enumitem}
\usepackage{calc}

\begin{document}

\begin{tcolorbox}[colback=black!10, colframe=black!50, title=Observações]
    \begin{itemize}
        \item Todas as páginas com resposta devem ter o nome e matrícula do
            aluno escritos com caneta no início (cabeçalho) ou no final
            (rodapé). Páginas que não obedeçam a esse critério não serão usadas
            na avaliação
    \end{itemize}
\end{tcolorbox}

\vspace{2em}

\begin{enumerate}
    \item \ratio{33} Considere uma placa plana de vidro de espessura \(L=\SI{1}{cm}\) e
        condutividade térmica \(k_v = \SI{0.78}{W/m\cdot\celsius}\). A superfície
        esquerda é mantida à temperatura \(T_0 = \SI{25}{\celsius}\), enquanto 
        a superfície direita permanece com temperatura \(T_L = \SI{10}{\celsius}\).
        Determine a densidade de fluxo de calor que atravessa a placa de vidro.

    \item \ratio{33} A parede plana de um forno industrial é constituída de uma camada
        de tijolos refratários de espessura \(L_1 = \SI{20}{cm}\) com condutividade 
        térmica \(k_1 = \SI{1.05}{W/m\cdot\celsius}\) e uma camada externa de um material 
        isolante com condutividade térmica \(k_2 = \SI{0.04}{W/m\cdot\celsius}\). O ar interno 
        é mantido com temperatura \(T_\text{ar,i} = \SI{800}{\celsius}\), constante,
        com coeficiente de transferência de calor por convecção \(h_i = \SI{30}{W/{m^2}\cdot\celsius}\).
        Considerando contato térmico perfeito entre os materiais sólidos, determine a 
        espessura \(L_2\) da camada de isolante para que a temperatura da superfície
        externa da parede do forno seja \(T_\text{e}= \SI{40}{\celsius}\) com uma perda
        de calor do forno para o ar ambiente de \(\SI{500}{W/m^2}\).

    \item \ratio{34} A figura abaixo mostra um esquema do fundo de uma cafeteira 
        elétrica constituído de uma parede plana, composta de uma chapa de aço com
        espessura \(L_A = \SI{2}{mm}\) e condutividade térmica \(k_A = \SI{40}{W/m\cdot\celsius}\),
        e de uma chapa de isolante com espessura \(L_I = \SI{4}{mm}\) e condutividade térmica
        \(k_I = \SI{0.06}{W/m\cdot\celsius}\). Entre as placas de aço e isolante há uma resistência 
        elétrica que dissipa uma potência de \SI{800}{W}. Considere a situação de regime
        permanente com a água em ebulição à temperatura \(T_\text{água}= \SI{100}{\celsius}\) e
        com coeficiente de transferência de calor por convecção \(h_\text{água}=\SI{3000}{W/{m^2}\cdot\celsius}\),
        enquanto o ar ambiente que está em contato com o isolante permanece com temperatura
        \(T_\text{ar} = \SI{25}{\celsius}\) e coeficiente de transferência de calor por
        convecção \(h_\text{ar} = \SI{10}{W/{m^2}\cdot\celsius} \). Considerando um contato térmico 
        perfeito entre a resistência elétrica e as chapas de aço e de isolante e que o fundo da cafeteira
        tem área \(A=\SI{0.018}{m^2}\), determine
        \begin{enumerate}
            \item temperatura \(T_R\) da resistência elétrica; e
            \item o fluxo de calor \(Q_\text{ar}\) perdido para o ar através da chapa
                isolante.

        \end{enumerate}

        \centering
        \begin{tikzpicture}
            \draw [fill=black!15]  (0,2) rectangle (10,3) node [midway] {Aço};
            \draw [fill=black!30] (0,1) rectangle (10,2) node [midway] {Resistência elétrica};
            \draw [fill=black!45] (0,0) rectangle (10,1) node [midway] {Isolante};
        \end{tikzpicture}
\end{enumerate}

\end{document}
