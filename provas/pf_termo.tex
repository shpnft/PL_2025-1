\documentclass[12pt,a4paper,brazilian]{article}

\usepackage{babel}
\usepackage[utf8]{inputenc}
\usepackage[T1]{fontenc}
\usepackage{lmodern}

\usepackage{amssymb,amsfonts,amsmath}

\usepackage{tikz}
\usetikzlibrary{calc,intersections}

%https://tex.stackexchange.com/a/100406
%29.7 cm - 1cm - 1cm - 144.90/28.4 cm = 22.60 cm
\usepackage[a4paper, totalheight=22.60cm,includeheadfoot,left=1.5cm, right=1.0cm, top=1cm]{geometry}
\setlength{\headheight}{144.90pt}

\setkeys{Gin}{keepaspectratio}

\newcommand{\cabeca}{
    \begin{tikzpicture}
        \node(Logo) {\includegraphics[width=2.5cm]{logo.png}};
        % \node(Logo) {\includegraphics[width=4.1cm]{logo.png}};

        \node(Local) at (Logo.north east) [anchor=north west, yshift=-0.25cm,
            align=center, execute at begin node=\setlength{\baselineskip}{3ex}]
            {
                \huge{\textbf{Universidade Federal do Amazonas}} \\
                \large{\textbf{Instituto de Ciências Exatas e Tecnologia}} \\
                \large{\textbf{\Description}}
            };

        \node(Ident) at (Local.south west) [anchor=north west, yshift=-0.25cm,
            align=left, execute at begin node=\setlength{\baselineskip}{2em}]
            {
                Professor: {\fontfamily{augie}\selectfont \Professor} \\
                Aluno(s):
            };
        % \draw [thick] (Logo.south west) -- ($(Logo.south west -| Local.south east)$);
        % \draw [red] (Logo.north west) rectangle (Logo.south east);
        % \draw [blue] (Local.north west) rectangle (Local.south east);
        % \draw [green] (Ident.north west) rectangle (Ident.south east);
    \end{tikzpicture}
}

\usepackage{fancyhdr}
\fancyhead{}
\fancyfoot{}
\fancyhead[c]{\cabeca}
\fancyfoot[r]{\fontfamily{augie}\selectfont Boa sorte!}

\pagestyle{fancy}
\renewcommand{\headrulewidth}{0pt}
\renewcommand{\footrulewidth}{0pt}

\newcommand{\ratio}[1]{(#1\% da nota)}
%-----------------------------------CUT HERE-----------------------------------

\def\Description{Termodinâmica -- Prova Final}
\def\Professor{Rodrigo de Farias Gomes}

\usepackage{siunitx}
\sisetup{locale = FR}

\usepackage{tcolorbox}
\tcbset{boxrule=0pt, top=0pt, bottom=0pt}

\DeclareMathOperator{\sen}{sen}
\DeclareMathOperator{\tg}{tg}
\usepackage{enumitem}

\begin{document}

\begin{tcolorbox}[colback=black!10, colframe=black!50, title=Observações]
    \begin{itemize}
        \item Todas as páginas com resposta devem ter o nome e matrícula do
            aluno escritos com caneta no início (cabeçalho) ou no final
            (rodapé). Páginas que não obedeçam a esse critério não serão usadas
            na avaliação
    \end{itemize}
\end{tcolorbox}

\vspace{2em}

\begin{enumerate}
    \item\ratio{30} A equação de estado de um certo gás é 
        \[
            (P+b)v = RT
        \]
        e sua energia interna específica é dada por 
        \[
            u=aT+bv+u_0
        \]
        \begin{enumerate}
            \item Encontre \(c_v\)
            \item Mostre que \(c_P -c_v = R\)
        \end{enumerate}
    \item\ratio{30} Em um só diagrama \(T-S\), esboce curvas para os seguintes processos
        reversíveis para um gás ideal, começando do mesmo estado inicial: (a)
        uma expansão isotérmica, (b) uma compressão adiabática, (c) uma expansão
        isobárica e (d) um processo isocórico em que é adicionado calor.

    \item\ratio{40} Uma massa \(m\) de um líquido a uma temperatura \(T_1\) é misturada
        com uma massa \(3m\) do mesmo líquido a uma temperatura \(T_2\).
        O sistema está termicamente isolado. Mostre que a variação de entropia do
        universo é
        \[
            4 m c_P \ln\frac{\tau+3}{4\tau^{1/4}}
        \]
        onde \(\tau=T_1/T_2\).


\end{enumerate}

\end{document}
