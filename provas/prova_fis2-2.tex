\documentclass[12pt,a4paper,brazilian, fleqn]{article}

\usepackage{babel}
\usepackage[utf8]{inputenc}
\usepackage[T1]{fontenc}
\usepackage{lmodern}

\usepackage{amssymb,amsfonts,amsmath}

\usepackage{tikz}
\usetikzlibrary{calc,intersections}

%https://tex.stackexchange.com/a/100406
%29.7 cm - 1cm - 1cm - 144.90/28.4 cm = 22.60 cm
\usepackage[a4paper, totalheight=22.60cm,includeheadfoot,left=1.5cm, right=1.0cm, top=1cm]{geometry}
\setlength{\headheight}{144.90pt}

\setkeys{Gin}{keepaspectratio}

\newcommand{\cabeca}{
    \begin{tikzpicture}
        \node(Logo) {\includegraphics[width=2.5cm]{logo.png}};
        % \node(Logo) {\includegraphics[width=4.1cm]{logo.png}};

        \node(Local) at (Logo.north east) [anchor=north west, yshift=-0.25cm,
            align=center, execute at begin node=\setlength{\baselineskip}{3ex}]
            {
                \huge{\textbf{Universidade Federal do Amazonas}} \\
                \large{\textbf{Instituto de Ciências Exatas e Tecnologia}} \\
                \large{\textbf{\Description}}
            };

        \node(Ident) at (Local.south west) [anchor=north west, yshift=-0.25cm,
            align=left, execute at begin node=\setlength{\baselineskip}{2em}]
            {
                Professor: {\fontfamily{augie}\selectfont \Professor} \\
                Aluno(s):
            };
        % \draw [thick] (Logo.south west) -- ($(Logo.south west -| Local.south east)$);
        % \draw [red] (Logo.north west) rectangle (Logo.south east);
        % \draw [blue] (Local.north west) rectangle (Local.south east);
        % \draw [green] (Ident.north west) rectangle (Ident.south east);
    \end{tikzpicture}
}

\usepackage{fancyhdr}
\fancyhead{}
\fancyfoot{}
\fancyhead[c]{\cabeca}
\fancyfoot[r]{\fontfamily{augie}\selectfont Boa sorte!}

\pagestyle{fancy}
\renewcommand{\headrulewidth}{0pt}
\renewcommand{\footrulewidth}{0pt}

\newcommand{\ratio}[1]{(#1\% da nota)}
%-----------------------------------CUT HERE-----------------------------------

\def\Description{Física II -- Prova 2}
\def\Professor{Rodrigo de Farias Gomes}

\usepackage{siunitx}
\sisetup{locale = FR}

\usepackage{tcolorbox}
\tcbset{boxrule=0pt, top=0pt, bottom=0pt}

\DeclareMathOperator{\sen}{sen}
\DeclareMathOperator{\tg}{tg}
\usepackage{enumitem}

\begin{document}

\begin{tcolorbox}[colback=black!10, colframe=black!50, title=Observações]
    \begin{itemize}
        \item Todas as páginas com resposta devem ter o nome e matrícula do
            aluno escritos com caneta no início (cabeçalho) ou no final
            (rodapé). Páginas que não obedeçam a esse critério não serão usadas
            na avaliação
    \end{itemize}
\end{tcolorbox}

\vspace{2em}

\begin{enumerate}

    \item \ratio{25} Uma onda senoidal de \SI{500}{Hz} se propaga em uma corda
        a \SI{250}{m/s}. (a) Qual é a distância entre dois pontos da corda cuja
        diferença de fase é \(\SI[parse-numbers=false]{\pi/4}{rad}\)? (b) Qual
        é a diferença de fase entre dois deslocamentos de um ponto da corda que
        acontecem com um intervalo de \SI{0.50}{ms}?

    \item \ratio{25} Se o período de um pêndulo simples de \SI{70.0}{cm} de
        comprimento é \SI{1.71}{s}, qual é o valor de \(g\) no local onde ele
        se encontra?

    \item \ratio{25} Duas ondas harmônicas propagam-se em uma corda no mesmo
        sentido, ambas com uma frequência de \SI{100}{Hz}, um comprimento de
        onda de \SI{2.0}{cm} e uma amplitude de \SI{2.0}{cm}. Ademais, elas se
        sobrepõem. Qual é a amplitude da onda resultante se as originais
        diferem de fase por (a) \(\pi/5\) e (b) \(\pi/4\)?

    \item \ratio{25} O ponto de ebulição do oxigênio, a \SI{1.00}{atm}, é
        \SI{90.2}{K}. Qual é o ponto de ebulição do oxigênio a \SI{1.00}{atm}
        nas escalas Celsius e Fahrenheit?
\end{enumerate}

\end{document}
