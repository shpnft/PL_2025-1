\documentclass[12pt,a4paper,brazilian]{article}

\usepackage{babel}
\usepackage[utf8]{inputenc}
\usepackage[T1]{fontenc}
\usepackage{lmodern}

\usepackage{amssymb,amsfonts,amsmath}

\usepackage{tikz}
\usetikzlibrary{calc,intersections}

%https://tex.stackexchange.com/a/100406
%29.7 cm - 1cm - 1cm - 144.90/28.4 cm = 22.60 cm
\usepackage[a4paper, totalheight=22.60cm,includeheadfoot,left=1.5cm, right=1.0cm, top=1cm]{geometry}
\setlength{\headheight}{144.90pt}

\setkeys{Gin}{keepaspectratio}

\newcommand{\cabeca}{
    \begin{tikzpicture}
        \node(Logo) {\includegraphics[width=2.5cm]{logo.png}};
        % \node(Logo) {\includegraphics[width=4.1cm]{logo.png}};

        \node(Local) at (Logo.north east) [anchor=north west, yshift=-0.25cm,
            align=center, execute at begin node=\setlength{\baselineskip}{3ex}]
            {
                \huge{\textbf{Universidade Federal do Amazonas}} \\
                \large{\textbf{Instituto de Ciências Exatas e Tecnologia}} \\
                \large{\textbf{\Description}}
            };

        \node(Ident) at (Local.south west) [anchor=north west, yshift=-0.25cm,
            align=left, execute at begin node=\setlength{\baselineskip}{2em}]
            {
                Professor: {\fontfamily{augie}\selectfont \Professor} \\
                Aluno(s):
            };
        % \draw [thick] (Logo.south west) -- ($(Logo.south west -| Local.south east)$);
        % \draw [red] (Logo.north west) rectangle (Logo.south east);
        % \draw [blue] (Local.north west) rectangle (Local.south east);
        % \draw [green] (Ident.north west) rectangle (Ident.south east);
    \end{tikzpicture}
}

\usepackage{fancyhdr}
\fancyhead{}
\fancyfoot{}
\fancyhead[c]{\cabeca}
\fancyfoot[r]{\fontfamily{augie}\selectfont Boa sorte!}

\pagestyle{fancy}
\renewcommand{\headrulewidth}{0pt}
\renewcommand{\footrulewidth}{0pt}

\newcommand{\ratio}[1]{(#1\% da nota)}
%-----------------------------------CUT HERE-----------------------------------

\def\Description{Termodinâmica -- Prova 2}
\def\Professor{Rodrigo de Farias Gomes}

\usepackage{siunitx}
\sisetup{locale = FR}

\usepackage{tcolorbox}
\tcbset{boxrule=0pt, top=0pt, bottom=0pt}

\DeclareMathOperator{\sen}{sen}
\DeclareMathOperator{\tg}{tg}
\usepackage{enumitem}

\begin{document}

\begin{tcolorbox}[colback=black!10, colframe=black!50, title=Observações]
    \begin{itemize}
        \item Todas as páginas com resposta devem ter o nome e matrícula do
            aluno escritos com caneta no início (cabeçalho) ou no final
            (rodapé). Páginas que não obedeçam a esse critério não serão usadas
            na avaliação

        \item Para evitar excesso de informação, optou-se por não
            descrever a notação usada. Caso haja
            dúvidas, pergunte ao professor.
    \end{itemize}
\end{tcolorbox}

\vspace{2em}

\begin{enumerate}
    \item \ratio{25} Mostre que a função de Helmholtz específica para um gás ideal, escolhendo 
        \(T\) e \(v\) como variáveis independentes, é
        \[
            f=c_v(T-T_0) -c_v T \ln \frac{T}{T_0} - 
            RT\ln\frac{v}{v_0}-s_0(T-T_0) + f_0
        \]

    \item \ratio{25} Mostre que a função de Helmholtz específica para um gás de van der Waals
        \[
            \left(P+\frac{a}{v^2}\right)(v-b)=RT
        \]
        escolhendo \(T\) e \(v\) como variáveis independentes, é
        \[
            f=c_v (T-T_0) - c_v \ln \frac{T}{T_0} - a\left(\frac{1}{v}-
            \frac{1}{v_0}\right)-
            RT\ln\left(\frac{v-b}{v_0-b}\right)-s_0(T-T_0)+f_0
        \]

    \item \ratio{25} A função de Gibbs específica de um gás é dada por
        \[
            g = -RT\ln(v/v_0) + Bv
        \]
        onde \(B\) é uma função de \(T\) somente. (a) Mostre explicitamente
        que esta forma da função de Gibss não especifica completamente as 
        propriedades do gás. (b) Que outra informação é necessária para que as 
        propriedades do gás possam ser completamente especificadas?

    \item \ratio{25} Mostre que a energia interna de um sistema a volume e entropia
        constantes deve  diminuir em qualquer processo espontâneo.

\end{enumerate}

\newpage
\def\Description{Termodinâmica -- Gabarito}

\begin{enumerate}
    \item Temos que
        \[
            f=u-Ts
        \]
        As equações 6-48
        \[
            u=c_v(T-T_0) +u_0
        \]
        e 6-45
        \[
            s=c_v\ln\frac{T}{T_0}+R\ln\frac{v}{v_0} + s_0
        \]
        do livro texto podem ser combinadas resultando em
        \[
            f=c_v(T-T_0) +u_0 - 
            c_vT\ln\frac{T}{T_0}-RT\ln\frac{v}{v_0} - s_0T
        \]
        Definindo \(f_0 = u_0-T_0s_0\), temos finalmente
        \[
            f=c_v(T-T_0) - 
            c_vT\ln\frac{T}{T_0}-RT\ln\frac{v}{v_0} - s_0(T-T_0) +f_0
        \]

    \item Temos que
        \[
            f=u-Ts
        \]
        As equações 6-51
        \[
            u=c_v(T-T_0)-a\left(\frac{1}{v}-\frac{1}{v_0}\right)+ u_0
        \]
        e 6-50 
        \[
            s=c_v\ln\frac{T}{T_0}+R\ln\left(\frac{v-b}{v_0-b}\right)+s_0
        \]
        do livro texto podem ser combinadas resultando em
        \[
            f=
            c_v(T-T_0)-a\left(\frac{1}{v}-\frac{1}{v_0}\right)+ u_0 -
            c_vT\ln\frac{T}{T_0}-RT\ln\left(\frac{v-b}{v_0-b}\right)-s_0T
        \]
        Definindo \(f_0 = u_0-T_0s_0\), temos finalmente
        \[
            f=
            c_v(T-T_0)-a\left(\frac{1}{v}-\frac{1}{v_0}\right) -
            c_vT\ln\frac{T}{T_0}-RT\ln\left(\frac{v-b}{v_0-b}\right)-s_0(T-T_0)+f_0
        \]

    \item 

\end{enumerate}

\end{document}
